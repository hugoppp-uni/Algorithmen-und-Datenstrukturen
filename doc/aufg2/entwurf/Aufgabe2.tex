% Package
\documentclass[11pt]{article}

\usepackage{amsmath}
\usepackage{cite}
\usepackage{graphicx}
\usepackage[utf8]{inputenc}
\usepackage[T1]{fontenc}
\usepackage{lmodern}
\usepackage[ngerman]{babel}
\usepackage{hyphenat}

\title{ADP Aufgabe 2 Entwurf, Abgabe 1}
\author{Team 1\\Hugo Protsch, Justin Hoffmann}

% Document
\begin{document}

    \maketitle

    \tableofcontents

    \newpage

    \section{Formales}\label{sec:Formales}

    %! suppress = MissingLabel

    \subsection*{Aufgabenaufteilung}
    Der Entwurf für Insertion Sort wurde zusammen entwickelt.\\
    Der Entwurf für Quick Sort wurde von Hugo entwickelt.\\
    Der Entwurf für Heap Sort wurde von Justin entwickelt.
    %! suppress = MissingLabel

    \subsection*{Quellenangaben}
    Es wurden lediglich Vorlesungsmaterialien verwendet.
    %! suppress = MissingLabel

    \subsection*{Bearbeitungszeitraum}
    Der gesamte Arbeitsaufwand für den Entwurf belief sich auf ca. 8 Stunden.
    %TODO Heapsort

    %! suppress = MissingLabel

    \subsection*{Aktueller Stand}
    %! suppress = MissingLabel

    \subsection*{Änderungen des Entwurfes}
    -- nicht zutreffend --



    \section{Insertion Sort}\label{sec:insertion-sort}

    \subsection{Algorithmus}\label{subsec:Ialgorithmus}
    Siehe Abbildung~\ref{fig:insertionS}.
    Bei Insertion Sort wird eine Liste durch das Einfügen von Elementen aus
    einem unsortierten Bereich (zunächst die komplette List) in einen sortieren
    Bereich (zunächst leer, <N1>) sortiert.

    Bei dem Einfügen eines Elements E in den sortierten Bereich muss dabei
    jeweils der Bereich bis zu dem Element durchlaufen werden, hinter das das
    Element E eingefügt werden muss (Siehe \frqq Insert into sorted list\flqq
    Subgraph).

    Somit wird der sortierte Bereich mit jeder Iteration um 1 erhöht <E1>.
    Sobald der sortierte Bereich alle Elemente enthält, wird die Liste
    zurückgegeben.
    Bei der Implementation des Algorithmus auf einfach verkettete Listen
    kann der sortierte und unsortierte Bereich getrennt werden.

    \subsection{Laufzeit}\label{subsec:Ilaufzeit}

    \begin{figure}[hbt]
        \caption{Insertion Sort}
        \centering
        \includegraphics[width = 8cm]{insertionS}\label{fig:insertionS}
    \end{figure}


    \section{Quick Sort}\label{sec:quick-sort}

    \subsection{Algorithmus}\label{subsec:Qalgorithmus}
    Siehe Abbildung~\ref{fig:qsort}.
    Bei Quick Sort wird zunächst willkürlich ein Pivot-Element aus der List
    ausgewählt.

    Anschließend wird die Liste in zwei Teile aufgeteilt: Die Liste L hält alle
    Elemente, die kleiner als das Pivot-Element sind, die Liste R alle
    Elemente, die größer als das Pivot-Element sind.
    Dafür wird die Liste elementweise durchlaufen und jedes Element mit dem
    Pivot-Element verglichen.
    Die Reihenfolge der Elemente in den Listen spielt keine Rolle.
    Das Pivot-Element selber kommt nicht in den Listen vor.

    Da alle Elemente, die kleiner als das Pivot-Element sind und alle Elemente,
    die größer als das Pivot-Element sind, nun getrennt vorliegen, ist die
    Position des Pivots eindeutig als zwischen den Listen L und R bestimmt.
    Somit kann der Algorithmus rekursiv auf die jeweiligen Listen erneut
    angewandt werden.

    Das Pivot-Element wird an die Liste R vorangestellt.
    Die Liste L wird der Liste R, inklusive Pivot, vorangestellt.
    Die Liste ist nun sortiert.

    \begin{figure}[hbt]
        \caption{Insertion Sort}
        \centering
        \includegraphics[width = 8cm]{qsort.pdf}\label{fig:qsort}
    \end{figure}

    \subsection{Laufzeit}\label{subsec:Qlaufzeit}


    \section{Heap Sort}\label{sec:heap-sort}

    \subsection{Algorithmus}\label{subsec:Halgorithmus}
    Siehe Abbildung~\ref{fig:hsort}.


    \begin{figure}[hbt]
        \caption{Insertion Sort}
        \centering
        \includegraphics[width = 8cm]{qsort.pdf}\label{fig:hsort}
    \end{figure}

    \subsection{Laufzeit}\label{subsec:Hlaufzeit}

\end{document}
