% Package
\documentclass[11pt]{article}

\usepackage{amsmath}
\usepackage{cite}
\usepackage{graphicx}
\usepackage[utf8]{inputenc}
\usepackage[T1]{fontenc}
\usepackage{lmodern}

\title{ADP Aufgabe 1, Abgabe 3}
\author{Team 1\\Hugo Protsch, Justin Hoffmann}

% Document
\begin{document}

    \maketitle


    \section{Formales}\label{sec:Formales}

    %! suppress = MissingLabel

    \subsection{Aufgabenaufteilung}
    Der Code wurde zusammen entwickelt.
    %! suppress = MissingLabel

    \subsection{Quellenangaben}

    Es wurden lediglich Vorlesungsmaterialien verwendet.

    %! suppress = MissingLabel

    \subsection{Bearbeitungszeitraum}
    Für die Bearbeitung und Überarbeitung des Entwurfs haben wir in etwa 10 bis
    12 Stunden benötigt.
    Für die Entwicklung des Quellcodes und die Laufzeitanalyse haben wir in
    etwa 15 bis 20 Stunden benötigt.
    %! suppress = MissingLabel

    \subsection{Aktueller Stand}
    Der Quellcode ist funktionsfähig und wurde auf Laufzeit überprüft.

    %! suppress = MissingLabel

    \subsection{Änderungen des Entwurfes}
    -- nicht zutreffend --


    \section{Laufzeitmessung}\label{sec:laufzeitmessung}

    \subsection{Zufällig}\label{subsec:zufaellig}

    \begin{center}
        \includegraphics[width=0.9\columnwidth] {ZeitAvg.pdf}
    \end{center}

    \subsubsection{Einstellungen}
    Die Folgenden Einstellungen wurden bei der Messung verwendet:
    \begin{itemize}
        \item 0 Startelemente
        \item 1000 Elemente Schrittgröße
        \item 20 Schritte
        \item Mitteln über 5 Messungen
    \end{itemize}
    
    \subsubsection{Ergebnisse}
        \paragraph{InsertBT}
        
        Bei InsertBT ist die längste Laufzeit zu erkennen, da wir den Baum zum Anhängen immer bis zum Ende durchlaufen müssen. Außerdem ist bei einer höheren Elementanzahl ein Anstieg der Steigung zu erkennen. Dies könnte auf eine logarithmische Laufzeit hindeuten, wie bereits im Entwurf von uns erwartet.

        \paragraph{DeleteBT}  
        
        DeleteBT hat bei zufälligen Elementen eine dezent kürzere Laufzeit als InsertBT, da wir zum Entfernen eines Elementes nicht zwangsläufig bis zum Ende Laufen müssen. Falls jedoch im Knoten des zu löschenden Elements sowohl ein linker, als auch einen rechter Teilbaum existiert, muss der Baum bis zum größten Element des linken Teilbaums traversiert werden, um das Ersatzelement zu finden. Auch hier lässt sich bei größerer Elementanzahl ein leichter Anstieg der Steigung erkennen. Dies entspricht ebenfalls den Erwartungen einer logarithmischen Komplexität.

        \paragraph{FindBT}
        
        FindBT hat eine deutlich kürzere Laufzeit als DeleteBT und InsertBT, da der Baum in jedem Fall nur bis zum gesuchten Element traversiert wird. Die Laufzeit-Komplexität.

        \paragraph{EqualBT}

        \paragraph{IsBT}

    \subsection{Aufsteigend und Absteigend}\label{subsec:average}

    \begin{center}
        \includegraphics[width=0.9\columnwidth] {ZeitAb.pdf}
    \end{center}
    \begin{center}
        \includegraphics[width=0.9\columnwidth] {ZeitAuf.pdf}
    \end{center}

    \subsubsection{Einstellungen}
    Die Folgenden Einstellungen wurden bei der Messung verwendet:
    \begin{itemize}
        \item 0 Startelemente
        \item 10000 Elemente Schrittgröße
        \item 50 Schritte
        \item Mitteln über 10 Messungen
    \end{itemize}

    \subsubsection{Ergebnisse}
    \begin{itemize}
        \item InsertBT:

        \item DeleteBT

        \item FindBT:

        \item EqualBT:

        \item IsBT:
    \end{itemize}


\end{document}
